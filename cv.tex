\documentclass[10pt,a4paper,ragged2e,withhyper]{altacv}
%% AltaCV uses the fontawesome5 and simpleicons packages.
%% See http://texdoc.net/pkg/fontawesome5 and http://texdoc.net/pkg/simpleicons for full list of symbols.

% Change the page layout if you need to
\geometry{left=1.25cm,right=1.25cm,top=1.5cm,bottom=1.5cm,columnsep=1.2cm}

% The paracol package lets you typeset columns of text in parallel
\usepackage{paracol}


% Change the font if you want to, depending on whether
% you're using pdflatex or xelatex/lualatex
% WHEN COMPILING WITH XELATEX PLEASE USE
% xelatex -shell-escape -output-driver="xdvipdfmx -z 0" mmayer.tex
\iftutex
  % If using xelatex or lualatex:
  \setmainfont{Lato}
\else
  % If using pdflatex:
  \usepackage[default]{lato}
\fi

% Change the colours if you want to
\definecolor{VividPurple}{HTML}{3E0097}
\definecolor{SlateGrey}{HTML}{2E2E2E}
     \definecolor{LightGrey}{HTML}{666666}
% \colorlet{name}{black}
% \colorlet{tagline}{PastelRed}
\colorlet{heading}{VividPurple}
\colorlet{headingrule}{VividPurple}
% \colorlet{subheading}{PastelRed}
\colorlet{accent}{VividPurple}
\colorlet{emphasis}{SlateGrey}
\colorlet{body}{LightGrey}

% Change some fonts, if necessary
% \renewcommand{\namefont}{\Huge\rmfamily\bfseries}
% \renewcommand{\personalinfofont}{\footnotesize}
% \renewcommand{\cvsectionfont}{\LARGE\rmfamily\bfseries}
% \renewcommand{\cvsubsectionfont}{\large\bfseries}

% Change the bullets for itemize and rating marker
% for \cvskill if you want to
\renewcommand{\cvItemMarker}{{\small\textbullet}}
\renewcommand{\cvRatingMarker}{\faCircle}
% ...and the markers for the date/location for \cvevent
% \renewcommand{\cvDateMarker}{\faCalendar*[regular]}
% \renewcommand{\cvLocationMarker}{\faMapMarker*}


% If your CV/résumé is in a language other than English,
% then you probably want to change these so that when you
% copy-paste from the PDF or run pdftotext, the location
% and date marker icons for \cvevent will paste as correct
% translations. For example Spanish:
% \renewcommand{\locationname}{Ubicación}
% \renewcommand{\datename}{Fecha}


%% Use (and optionally edit if necessary) this .tex if you
%% want to use an author-year reference style like APA(6)
%% for your publication list
% \input{pubs-authoryear.tex}

%% Use (and optionally edit if necessary) this .tex if you
%% want an originally numerical reference style like IEEE
%% for your publication list
\usepackage[backend=biber,style=ieee,sorting=ydnt,defernumbers=true]{biblatex}
%% For removing numbering entirely when using a numeric style
\setlength{\bibhang}{1.25em}
\DeclareFieldFormat{labelnumberwidth}{\makebox[\bibhang][l]{\itemmarker}}
\setlength{\biblabelsep}{0pt}
\defbibheading{pubtype}{\cvsubsection{#1}}
\renewcommand{\bibsetup}{\vspace*{-\baselineskip}}
\AtEveryBibitem{%
  \iffieldundef{doi}{}{\clearfield{url}}%
}


%% sample.bib contains your publications
\addbibresource{sample.bib}

\begin{document}
\name{RASHED AZAD}
\tagline{\textbf{Embedded Software Developer}}
% Cropped to square from https://en.wikipedia.org/wiki/Marissa_Mayer#/media/File:Marissa_Mayer_May_2014_(cropped).jpg, CC-BY 2.0
%% You can add multiple photos on the left or right
% \photoR{2.5cm}{mmayer-wikipedia-cc-by-2_0}
% \photoL{2cm}{Yacht_High,Suitcase_High}
\personalinfo{
	\makebox[\linewidth][l]{%
		% Not all of these are required!
		% You can add your own with \printinfo{symbol}{detail}
		\phone{+49(0)15208539926}  % Add your phone number here
		\email{rashed.azad@gmail.com}
		% \phone{000-00-0000}
		% \mailaddress{Address, Street, 00000 County}
		% \homepage{marissamayr.tumblr.com}
		% \twitter{@marissamayer}
		% \xtwitter{@marissamayer}
		\github{rashed-azad} % I'm just making this up though.
		\linkedin{rashed-azad-jr}
		\location{Metzingen, Germany}
		% \orcid{0000-0000-0000-0000} % Obviously making this up too.
		%% You can add your own arbitrary detail with
		%% \printinfo{symbol}{detail}[optional hyperlink prefix]
		% \printinfo{\faPaw}{Hey ho!}
		%% Or you can declare your own field with
		%% \NewInfoFiled{fieldname}{symbol}[optional hyperlink prefix] and use it:
		% \NewInfoField{gitlab}{\faGitlab}[https://gitlab.com/]
		% \gitlab{your_id}
		%%
		%% For services and platforms like Mastodon where there isn't a
		%% straightforward relation between the user ID/nickname and the hyperlink,
		%% you can use \printinfo directly e.g.
		% \printinfo{\faMastodon}{@username@instace}[https://instance.url/@username]
		%% But if you absolutely want to create new dedicated info fields for
		%% such platforms, then use \NewInfoField* with a star:
		% \NewInfoField*{mastodon}{\faMastodon}
		%% then you can use \mastodon, with TWO arguments where the 2nd argument is
		%% the full hyperlink.
		% \mastodon{@username@instance}{https://instance.url/@username}
	}
}

\makecvheader

%% Depending on your tastes, you may want to make fonts of itemize environments slightly smaller
\AtBeginEnvironment{itemize}{\small}

%% Set the left/right column width ratio to 6:4.
\columnratio{0.6}

% Start a 2-column paracol. Both the left and right columns will automatically
% break across pages if things get too long.
\begin{paracol}{2}

	\cvsection{Professional Summary}

	{\small
	Embedded Software Developer with 5+ years of experience in real-time and system-critical applications,
	particularly in robotics and embedded Linux environments. Proven expertise in modern C++ and Rust, with a
	strong focus on performance, scalability, and algorithmic efficiency. Currently based in Germany and actively
	seeking long-term roles in the country’s high-tech and industrial innovation sectors.
	}

	\cvsection{Experience}

	\cvevent{Embedded Software Developer}{Nova Measuring Instruments GmbH}{January 2023 -- Ongoing}{Bad Urach, Germany}
	\begin{itemize}
		\item Developed applications using modern C++ and Rust, with a focus on safety, concurrency, and performance.
		\item Designed scalable architectures for embedded Linux (ARM-based) platforms.
		\item Delivered optimized solutions using advanced algorithms, parallel programming, and distributed systems.
		\item Automated workflows and scripts using Python.
		\item Worked in Agile/SCRUM teams using JIRA.
		\item Provided feasibility studies and architecture recommendations for new features.
	\end{itemize}

	\vspace{0.5em}
	{\color{accent}\textbf{\large Key Achievements}}
	\vspace{0.3em}
	\begin{itemize}
		\item Improved system performance through memory profiling and algorithmic optimizations.
		\item Led migration of legacy C code to modern C++, reducing technical debt and improving code maintainability.
	\end{itemize}

	\divider

	\cvevent{Software Developer}{Neura Robotics GmbH}{July 2020 -- December 2022}{Metzingen, Germany}
	\begin{itemize}
		\item Developed high-performance C++ code for real-time, safety-critical robotic systems.
		\item Designed and implemented interfaces between C++ backends and GUIs.
		\item Integrated and maintained modular software architectures in evolving systems.
		\item Analyzed, debugged, and optimized components in live environments.
		\item Created efficient algorithms and transformed complex logic into robust architectures.
		\item Collaborated in cross-functional teams across hardware and software.
		\item Used Git and JIRA for source control and project tracking.
	\end{itemize}

	% \divider

	% \cvevent{Product Engineer}{Google}{23 June 1999 -- 2001}{Palo Alto, CA}

	% \begin{itemize}
	% \item Joined the company as employe \#20 and female employee \#1
	% \item Developed targeted advertisement in order to use user's search queries and show them related ads
	% \end{itemize}

	%% Switch to the right column. This will now automatically move to the second
	%% page if the content is too long.
	\switchcolumn

	% Don't overuse these \cvtag boxes — they're just eye-candies and not essential. If something doesn't fit on a single line, it probably works better as part of an itemized list (probably inlined itemized list), or just as a comma-separated list of strengths.

	% The `ragged2e` document class option might cause automatic linebreaks between \cvtag to fail.
	% Either remove the ragged2e option; or 
	% add \LaTeXraggedright in the paragraph for these \cvtag
	
	
	\cvsection{Skills}

	\cvevent{Programming Languages}{}{}{}
	{\LaTeXraggedright
		\cvtag{Modern C++}
		\cvtag{Rust}
		\cvtag{Python}
		\cvtag{Embedded C}
		\par}

	\vspace{1.2em}

	\cvevent{Systems}{}{}{}
	{\LaTeXraggedright
		\cvtag{Embedded Linux (ARM)}
		\par}

	\vspace{1.2em}

	\cvevent{Communication Protocols}{}{}{}
	{\LaTeXraggedright
		\cvtag{I2C}
		\cvtag{CAN}
		\par}

	\vspace{1.2em}

	\cvevent{Development Tools}{}{}{}
	{\LaTeXraggedright
		\cvtag{Git}
		\cvtag{CMake}
		\cvtag{Docker}
		\cvtag{Google Test}
		\par}

	\vspace{1.2em}

	\cvevent{Specialties}{}{}{}
	{\LaTeXraggedright
		\cvtag{Memory Management}
		\cvtag{Secure Coding}
		\cvtag{Algorithm Design}
		\cvtag{System Optimization}
		\par}

	\vspace{1.2em}

	\cvevent{Best Practices}{}{}{}
	{\LaTeXraggedright
		\cvtag{Agile/SCRUM}
		\cvtag{CI/CD}
		\cvtag{Test-Driven Development}
		\par}

	\cvsection{Languages}

	\cvskill{English}{5}
	\cvskill{German}{3.5}
	\cvskill{Bengali}{5}

	\cvsection{Education}

	\cvevent{M.Sc.\ in Mechatronic}{University of Siegen}{2015 -- 2019}{Siegen, Germany}

	\vspace{0.5em}

	\cvevent{B.Sc.\ in Electrical \& Electronic Engineering}{American International University-Bangladesh}{2009 -- 2013}{Dhaka, Bangladesh}

	\newpage

\end{paracol}

\end{document}
